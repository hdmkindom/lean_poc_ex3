\documentclass[11pt]{article}
\usepackage[UTF8]{ctex}
\usepackage{amsmath,amssymb,amsthm}
\usepackage{geometry}
\geometry{a4paper, margin=2.5cm}

\title{命题证明及形式化拆解}
\author{hdmkingdom}
\date{\today}

% 定理环境定义
\newtheorem{theorem}{命题}
\newtheorem{lemma}{Lemma}
\newtheorem{remark}{注}
\newtheorem{definition}{定义}
\newtheorem{corollary}{推论}

\begin{document}

\maketitle

\section*{题目(Problem Statement)}

\subsection*{English}
Let $x_{0}=5$ and $x_{n+1}=x_{n}+\frac{1} {x_{n}} ~ ( n=0, 1, 2, \ldots)$ . 

Prove that $4 5 < x_{1 0 0 0} <$ 45.1

\subsection*{中文}
设 $x_{0}=5$,且 $x_{n+1}=x_{n}+\frac{1}{x_{n}} ~ (n=0, 1, 2, \ldots)$。

证明:$45 < x_{1000} < 45.1$。

\section*{证明(Proof Sketch)}

1. 两侧平方:

\begin{align}
x_{n+1}^2 &= x_n^2 + \frac{1}{x_n^2} + 2\\
x_{n+1}^2 - x_n^2 &= \frac{1}{x_n^2} + 2 \label{eq:formula2}
\end{align}

2.因为 $x_n > 0$,$x_{n+1} > x_n$,因此:

\begin{align}
x_{n+1}^2 & = x_0^2 + \sum_{k=0}^{n}(x_{k+1}^2 - x_k^2)\\
& = x_0^2 + \sum_{k=0}^{n}(\frac{1}{x_k^2} + 2) \\
& = 25 + 2(n+1) + \sum_{k=0}^{n}\frac{1}{x_k^2} \label{eq:formula5} 
\end{align}

3. 确定范围 取 $n = 999$

\begin{align}
x_{100}^2 & = 25 + 2000 + \sum_{k=0}^{999}\frac{1}{x_k^2}\\
& = 2025 + \sum_{k=0}^{999}\frac{1}{x_k^2} \\
& > 2025 \\
& = 45^2
\end{align}

则小值确定,又

\begin{align}
x_{100}^2 & = 2025 + \sum_{k=0}^{99}\frac{1}{x_k^2} + \sum_{k=100}^{999}\frac{1}{x_k^2} \\
& < 2025 + \frac{100}{x_0^2} + \frac{900}{x_100^2} \\
& < 2033 \\ 
& < 45.1^2
\end{align}

则大值确定

\qed


\section*{形式化证明步骤(Formal Lemma Decomposition)}

\begin{lemma}
    由原方程处理两侧平方部分,得到平方差式 $\eqref{eq:formula2}$
\end{lemma}

\begin{lemma}
    对 $x_{n+1}^2$ 做处理,得到 $\eqref{eq:formula5}$
\end{lemma}

\begin{lemma}
    通过 $n=999$ 的取值,确定 $x_{1000}^2$ 的范围,即得到 $(6) 与 (10)$
\end{lemma}

\begin{lemma}
    对 $(6)$ 做处理,得到 $x_{1000}^2 > 2025$,得到小值
\end{lemma}

\begin{lemma}
    对 $(10)$ 做处理,得到 $x_{1000}^2 < 2033$,得到大值
\end{lemma}

结束
\end{document}
